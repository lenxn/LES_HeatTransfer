Large Eddy Simulation, a subdomain of Computational Fluid Dynamics, is recently experiencing an increased attention, due to increasing capabilities of the necessary hardware, in detail CPU and memory. In most sectors it is not yet industrial standard, because of its height demand in terms of resources, but it will most likely become an important tool for the investigation of complex flow problems in near future. 

Therefore this bachleor's project compromises the execution of a high-resolution simulation of the heat transfer on a wing surface in three dimensions. The given geometry for this task is a NACA 0012 airfoil and the used software tools are Ansys ICEM and Ansys CFX. 

Subsequent the achieved results are compared to results obtained from a similar RANS simulation, which is nowadays standard for industrial application. Based on this evaluation the applicability of LES is scrutinized and discussed.