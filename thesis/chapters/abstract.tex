Turbulence is a phenomenon that occurs more or less in almost every natural flow. 
This leads to great ambitions in terms of calculating turbulent flows in order to predict their behavior.
The objective of this work is the investigation of the heat transfer on a NACA 0012 airfoil by means of the Large Eddy Simulation.
The LES Simulation has not yet become standard for industrial application, due to its high demand on resources.

Large Eddy Simulation, a subdomain of Computational Fluid Dynamics, is recently experiencing an increased attention, due to increasing capabilities of the necessary hardware, in detail CPU and memory. In most sectors it is not yet industrial standard, because of its heigh demand in terms of resources, but it will become an important tool for investigation of complex flow problems in near future. Therefore the aim of this Bachleors project is the execution of a high-resolution simulation of the heat transfer on a wing surface in three dimensions. The given geometry for this task is a NACA 0012 airfoil and the software used will be Ansys ICEM and Ansys CFX. Subsequent the achieved results shall be compared to results obtained from a RANS-simulation, which are nowadays standard for industrial application.

Due to the complexity of the Large Eddy Simulation a majority of the work was studying the theoretical basics as well as performing LES in practice in order to achieve the necessary skills.