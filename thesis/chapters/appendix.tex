\begin{lstlisting}
%%%%%%%%%%%%%%%%%%%%%%%%%%%%%%%%%%%%%%%%%%%%%%%%%%%%%%%%%%%%%%%%%%%%%%%%%%%
%
% Title:            cell_height.m
% Version:          2.2
% Author:           Stefan Lengauer
% Date:             16th February 2015
% Description:      File for computing the necessary cell height for the
%                   cells attached to the wing surface.
%
%%%%%%%%%%%%%%%%%%%%%%%%%%%%%%%%%%%%%%%%%%%%%%%%%%%%%%%%%%%%%%%%%%%%%%%%%%%

clear all;
close all;

% Definition of variables
rho = 1.168;    % Density, [kg m^-3]
U = 66.8;       % Velocity, [m/s]
L = 1;          % Characteristic length scale, [m]
mu = 18.48e-6;  % Dynamic viscosity [Pa*s]
yplus = 1;      % y+, dimensionless

Re = rho * U * L / mu;
Cf = 0.079 * power( Re, -0.25 );

Tau_w = 1/2 * Cf * rho * power( U, 2 );

Utau = sqrt( Tau_w / rho );
deltay = yplus * mu /( rho * Utau );
\end{lstlisting}
\newpage
\begin{lstlisting}
%%%%%%%%%%%%%%%%%%%%%%%%%%%%%%%%%%%%%%%%%%%%%%%%%%%%%%%%%%%%%%%%%%%%%%%%%%%
%
% Title:            wall_heat_flux_plot.m
% Version:          1.0
% Author:           Stefan Lengauer
% Date:             15th February 2015
% Required Files:   wall_heat_flux_stationary.csv
%                   wall_heat_flux_transient.csv
% Description:      Script for creating and saving the data plots
%                   obtained from CFX-Post.
%
%%%%%%%%%%%%%%%%%%%%%%%%%%%%%%%%%%%%%%%%%%%%%%%%%%%%%%%%%%%%%%%%%%%%%%%%%%%

clear all;
close all;

%% Data Import
STAT = csvread( '../simulation_data/wall_heat_flux_stationary.csv' );
TRANS = csvread( '../simulation_data/wall_heat_flux_transient.csv' );

x_stat = STAT( :, 1 );
y_stat = STAT( :, 4 );

x_trans = TRANS( 3:350, 1 );
y_trans = TRANS( 3:350, 4 );


%% Plot
hold on;
grid;

plot( x_stat, y_stat, 'linewidth', 2, 'color', 'blue' )
plot( x_trans, y_trans, 'linewidth', 2, 'color', 'red' )

axis( [0, 1, 0, 500] );
title( 'Wall Heat Flux on NACA 0012 Airfoil' )
legend( 'RANS', 'Large Eddy Simulation' )
xlabel( 'X [m]' )
ylabel( 'Wall Heat Flux [W m^-2]' )

%% Save Plot
saveas( figure(1), '../images/Wall_Heat_Flux_Plot.png', 'png' )
\end{lstlisting}
\newpage
\begin{lstlisting}
%%%%%%%%%%%%%%%%%%%%%%%%%%%%%%%%%%%%%%%%%%%%%%%%%%%%%%%%%%%%%%%%%%%%%%%%%%%
%
% Title:            dimensionless_coefficients.m
% Version:          1.3
% Author:           Stefan Lengauer
% Date:             3rd March 2015
% Description:      Script for computation for the dimensionless
%                   coefficients, necessary for the evaluation of the 
%                   results.
%
%%%%%%%%%%%%%%%%%%%%%%%%%%%%%%%%%%%%%%%%%%%%%%%%%%%%%%%%%%%%%%%%%%%%%%%%%%%

clear all;
close all;

% Simulation parameters

c = 1.0;                % Chord length, [m]
t = 12/100;             % Maximum profile height, [m]
w = 66.8;               % Fluid velocity, [m s^-1]  

tw = 26 + 273.15;       % Temperature at the wing surface, [K]
tf = 25 + 273.15;       % Temperature of the fluid, [K]
A = 0.5967;             % Wing surface, [m^2]
whf_trans = 257.0520;   % Wall heat flux at stagnation point from transient
                        % simulation, [W m^-2]
whf_stat = 253.6925;    % Wall heat flux at stagnation point from
                        % stationary simulation, [W m^-2]

% Material properties for air at 25C

cp = 1007;              % Heat transfer coefficient, [J kg^-1 K^-1]
eta = 18.48e-6;         % Dynamic viscosity, [kg m^-1 s^-1]
lambda = 26.06e-3;      % Thermal conductivity, [W K^-1 m^-1] 
ypsilon = 15.82e-6;     % Kinematic viscosity, [m^2 s^-1]



R_LE = 1.1019 * power( t, 2 );  % Radius Leading edge, [m]
l = R_LE * 2;                   % Characteristic length scale, [m]


% Reynolds number
Re = w * l / ypsilon;

%% Theoretical values according to the fluid properties

% Prandtl number
Pr_id = cp * eta / lambda;

% Nusselt number
Nu_id = 1.14 * power( Pr_id, 0.4 ) * power( Re, 0.5 );

% Froude number
Fr_id = Nu_id / power( Re, 0.5 );

% Heat transfer coefficient
alpha_id = Nu_id * lambda / l;


%% Values with the Heat Transfer coefficient obtained from the transient
% simulation

% Heat transfer coefficient
alpha_stat = whf_trans / ( tw - tf );

% Nusselt number
Nu_stat = alpha_stat * l / lambda;

% Froude number
Fr_trans = Nu_stat / power( Re, 0.5 );

%% Values with the Heat Transfer coefficient obtained from the stationary
% simulation

% Heat transfer coefficient
alpha_stat = whf_stat / ( tw - tf );

% Nusselt number
Nu_stat = alpha_stat * l / lambda;

% Froude number
Fr_stat = Nu_stat / power( Re, 0.5 );
\end{lstlisting}
\newpage
\begin{lstlisting}
%%%%%%%%%%%%%%%%%%%%%%%%%%%%%%%%%%%%%%%%%%%%%%%%%%%%%%%%%%%%%%%%%%%%%%%%%%%
%
% Title:            drag_coefficient.m
% Version:          1.3
% Author:           Stefan Lengauer
% Date:             13th February 2015
% Required Files:   force_x_18450.csv
%                   force_x_18460.csv
%                   force_x_18470.csv
%                   force_x_18480.csv
%                   force_x_18490.csv
%                   force_x_18500.csv
%                   force_x_18510.csv
%                   force_x_18520.csv
%                   force_x_18530.csv
%                   force_x_18540.csv
%                   force_x_18550.csv
% Description:      Script for computing the drag coefficient of the 
%                   airfoil for the last 100 timesteps.
%
%%%%%%%%%%%%%%%%%%%%%%%%%%%%%%%%%%%%%%%%%%%%%%%%%%%%%%%%%%%%%%%%%%%%%%%%%%%


rho = 1.1839;           % density of air at 25 degrees, [kg m^-3]
u = 66.8;               % inlet speed, [m s^-1]
max_thickness = 0.12;   % max thickness of the profile, [m]
width = 0.3;            % profile width, [m]


% initialization of the coefficient vector
CD = zeros( 1, 11 );

for i = 450:10:550
    file = strcat( '../simulation_data/force_x_18', int2str( i ), '.csv' );
    A = csvread( file );
    force_x = A(:,4);
    
    % computation of the drag coefficient
    index = ( i - 450 )/10 + 1;
    CD(index) = sum( force_x ) / ...
        ( 1/2 * rho * power( u, 2 ) * width * max_thickness );
end
\end{lstlisting}