As mentioned in the abstract the aim of this project is the conduction of a heat transfer by means of a large eddy simulation and afterwards comparing the obtained results with the results of a stationary simulation of the same flow problem and analyzing deviations and similarities.
\section{Investigation of the wall heat flux}
The core of this project is the investigation of the wall heat flux on the wing surface. The basis for this examination are the results obtained from the simulations, which are plotted in Figure 3.2 and the calculation results, belonging to them, in table ??. Altough in this plot it seems like there is just one graph for each simulation type, there are actually two for each - one for the upper side and one for the bottom side of the wing. However, due to the symetry of the geometry and the flow conditions their heat transfer along the profile is almost the same, appart from numerical inaccuracies.
	
In the plot of the stationary result there are heavy flunctuations visible at the front end of the airfoil. This is physically illogically and results most likely from the application of the SST (Shear-Stress Transport) turbulence model for this simulation. The LES results seem much more convincing in this respect and it can be oberved that they feature a much higher wall heat flux at the front section of the wing and a lower one at the rear section, while it is equal to the stationary simulation at about forty percent wing depth. This agrees with the exectations, because in a turbulent flow the heat transfer is much better than in a laminar flow, since the turbulent vortices movement favors the energy exchange.
\subsection{Interpretation of the dimensionless numbers}
This subchapter is dedicated to analysing of the dimensionless number refered to in table ??. The Reynolds number is of course the same for both solutions, since it is independent from heat transfer.
The parameter of interest is the Foude number, which is almost equal to one for a cylinder. For the transient solution the Froude number shows a deviation of about fourteen percent from this value. What causes this inaccuracy may be the subject of further investigations, but an interesting fact here is, that it is still closer to the desired result than the stationary simulation.
\subsection{Comparison Large Eddy Simulation and RANS equation}
As already mentioned the Large Eddy Simulation requires massive ressources and a very sophisticated mesh compared to the RANS equations. However there are significant reasons, why LES becomes more and more attractive than RANS. One major drawback of the RANS equations is, that they are not sufficiently reliable in terms of prediction of heat transfers. As it is the case with this simulation, where the RANS equations come up with a physically rather questionable behaviour of the heat transfer. 
Furthermore LES is capable of dealing with plenty of different flow conditions, without relying on a priori assumptions.

