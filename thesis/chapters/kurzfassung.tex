Large Eddy Simulation, ein Teilbereich der numerischen Strömungsmechanik, erfährt in letzter Zeit erhöhte Beachtung dank der steigendenden Leistungsfähigkeit der erforderlichen Hardware, insbesondere CPU und Speicher. In den meisten Bereichen ist sie aufgrund ihres hohen Ressourcenaufwandes noch nicht Industriestandart, aber in naher Zukunft wird sie ein wichtiges Instrument zur Untersuchung von komplexen Str\"omungsproblemen werden.

Aus diesem Grund beinhalted dieses Bachelorprojekt die Durchf\"uhrung einer hochauflösenden Simulation eines W\"arme\"ubergangs an einem dreidimensionalen Fl\"ugelprofil. Die, für diese Aufgabe gegebene Geometrie, ist ein NACA 0012 Fl\"ugelprofil und die verwendeten Softwarepakete beinhalten Ansys ICEM und Ansys CFX. 

Anschließend werden die erzielten Resultate mit den Resultaten einer vergleichbaren RANS Simulation verglichen, welche momentan Standart für industrielle Anwendungen sind. Diese Auswertung dient als Grundlage für die darauffolgende Untersuchung und Diskussion der Anwendbarkeit der Large Eddy Simulation.