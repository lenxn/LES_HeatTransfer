Der Inhalt dieser Arbeit umfasst die Simulation des Wärmeübergangs an einer Flügeloberfläche mithilfe des sogenannten Large Eddy Turbulenzmodells. Im Gegensatz zu den standartmäßig verwendeten RANS (Reynoldsgemittelten Navier Stokes) Modellen erfordert dieses Verfahren einen erhöhten Resourcenaufwand was die Berechnung betrifft. Mit zunehmender Leistungsfähigkeit von Computern, was CPU Leistung und verfügbarer Speicher betrifft gewinnt dieses Verfahren jedoch, immer mehr an Bedeutung für die Untersuchung industriell bedeutsamer Strömungsprobleme.
Im Zuge der Arbeit wird die Anwendbarkeit und Akkuratät dieses Verfahrens anhand einer einfachen Modellkonfiguration, dem NACA 0012 Profil durchgeführt. Anschließend wurden die Ergebnisse der Simulation mit den Ergebnissen der RANS Simulation an selbigem Modell verglichen. Ein Großteil der Projektarbeit bestand jedoch aus Aneignung der theoretischen Grundlagen, sowie Einarbeitung in die praktische Anwendung der Large Eddy Simulation.
